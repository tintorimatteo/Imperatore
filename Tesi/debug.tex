\chapter{Debug}
\label{chap:debug}
\section{hevc\_nvenc}
\label{sec:nvenc}
Il codec sovracitato, adottato e spiegato a inizio documentazione, non è stato adottato casualmente per disponibilità di questo encoder nella lista, ma per essere adottato è stato seguito un procedimento più ostico che merita una descrizione accurata. 
Innanzitutto la versione di \textbf{FFMPEG} adottata non è stata compilata con i flag adatti per una scelta di un encoder H.265, per cui abbiamo dovuto reinstallare la stessa in maniera differente: ciò verrà spiegato in seguito. Come prima cosa, nell'uso di questa libreria grazie all'istruzione \textbf{FFmpegWriter} abbiamo potuto constatare che vi era un errore bloccante che non veniva descritto con verbosità tale da capirne l'origine. A video l'errore si limitava a essere descritto come \emph{broken pipe} non appena veniva iniziata la scrittura del primo frame del video di cui si iniziava la codifica, anche se la libreria usata era "libx264". Per causa di ciò, dopo accurate ricerche, si è dovuto impostare un paramentro aggiuntivo nell'istruzione di creazione dell'oggetto \textbf{FFmpegWriter} scritto come \emph{verbosity=1}. A questo punto, grazie ad un log dell'errore con verbosità maggiore come segue: \emph{Unrecognized option 'height’} e \emph{Unknown decoder 'libx265'} si è potuto constatare che l'errore dipendeva da questi parametri e abbiamo così preso alcuni provvedimenti. Il parametro height è stato tolto, mentre per l'encoder scelto abbiamo testato se effettivamente questo non fosse disponibile con l'istruzione: \emph{ffmpeg -codecs } | \emph{grep 26} ed è risultato che il codec \textbf{hevc} esiste ma è utilizzabile solo per la decodifica.\\
Cercando nel canale loopbio abbiamo potuto confermare che si può attivare la codifica H.265 utilizzando un pacchetto del canale stesso, ma ciò è possibile disinstallando\\ \textbf{FFMPEG} e \textbf{x264,x265} con l'istruzione:
\emph{conda uninstall x264 x265 ffmpeg —force} e reinstallandoli dal canale loopbio con l'istruzione: \emph{conda install ffmpeg x264 x265 -c loopbio --force-reinstall}\\
Un problema aggiuntivo si è verificato in quanto la versione di x264 non è alla versione richiesta di \textbf{FFMPEG}, e l'errore se utilizzata l'istruzione \emph{ffmpeg -codecs } risulta: \emph{error while loading shared libraries: libx264.so.138: cannot open shared object file: No such file or directory.}\\
Di conseguenza, abbiamo dovuto cercare il package che fornisce la versione di x264 che \textbf{FFMPEG} desidera, con l'istruzione: \emph{ls -la ~/miniconda3/envs/tf\_cpu/lib/libx2*} e ciò è sfociato con l'istruzione: \emph{conda install -c lightsource2 x264}\\
Adesso ffmpeg -codecs | grep 26 riporta:\\
 DEV.L. hevc H.265 / HEVC (High Efficiency Video Coding)  (encoders: nvenc\_hevc hevc\_nvenc )
\\La Lettera "E" nelle iniziali "DEVL" significa che è possibile a questo punto effettuare una codifica, oltre alla decodifica che era già possibile inizialmente.\\
 A questo punto, è bastato sostituire il parametro \textbf{-vcodec} da \textbf{libx265} a \textbf{hevc\_nvenc} per far partire la codifica con l'encoder ottimale scelto.
