\chapter{Sistema di segmentazione oggetti}
\label{chap:preparation}
In questo capitolo, per iniziare, si vuole distinguere la differenza tra un training di una rete blackbox e un fine tuning di una rete preaddestrata.\\
Essendo \textbf{Mask-RCNN} una rete preaddestrata per circa 60 classi, si parlerà innanzitutto di un fine tuning e non di un allenamento di una rete partendo da zero esperienza(blackbox).\\
La cosa che distingue il nostro agire in questo caso è che faremo un fine tuning, non sulle circa 60 classi sulla quale è stata preaddestrata la rete, ma solamente su 2 classi che rappresenteranno i giocatori e la palla all'interno del campo di calcio.
\newpage
\section{Strutturazione Fine-Tuning}
Come premessa, bisogna dichiarare il fatto che per un buon fine tuning serve scegliere un software di labeling che fornisce le annotazioni in un certo formato, solitamente XML.
Nel nostro caso, siccome sappiamo che PASCALVOC XML viene usato quando c'è da fare object detection mentre noi cerchiamo di fare instance segmentation, abbiamo usato un software chiamato \textbf{labelme} che produce annotazioni con estensione JSON in modo da prelevare facilmente i dati dei poligoni racchiusi nei files.
Lo strumento \textbf{labelme} viene utilizzato per tracciare poligoni intorno ai giocatori nei frames, cioè le immagini singole di un video che è stato frammentato con uno strumento come ad esempio \textbf{Avidemux}. \\
Nel software \textbf{labelme} quando viene scelto un tracciamento della curva \emph{poligono}, questi poligoni devono essere tracciati come linea spezzata chiusa con un numero di punti adatto a circondare con la minima tolleranza possibile la forma di un giocatore, e nel caso della scelta sul software \textbf{labelme} della curva \emph{cerchio}, esso andrà a circondare la palla.
Chiaramente, un giocatore potrà essere all'interno del campo in diverse posizioni, ad esempio in piedi fermo, in piedi in corsa oppure a terra come dopo una parata o una scivolata; tutti questi casi devono essere circondati da punti che formano un poligono per far sì che la rete \textbf{MaskRCNN} abbia un allenamento soddisfacente.\\
Una volta salvato il frame modificato con i poligoni, il software \textbf{labelme} genererà un file JSON con all'interno i dati del nome dell'immagine, la sua altezza e larghezza, il numero e i punti dei poligoni in modo da essere letti successivamente da codice.
L'operazione di contornamento dei giocatori e della palla con i poligoni di punti, si chiama \textbf{labeling}, e si può effettuare un labeling anche massivo specificando all'apertura del software \textbf{labelme}  una directory dove sono presenti tutti i frames del video obbiettivo e in contempo una directory vuota dove produrre in massa i file JSON delle loro annotazioni; ogni file di annotazione viene prodotto non appena il labeling di un immagine viene ultimato e quindi l'immagine viene salvata.